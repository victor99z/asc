% Options for packages loaded elsewhere
\PassOptionsToPackage{unicode}{hyperref}
\PassOptionsToPackage{hyphens}{url}
%
\documentclass[
]{article}
\usepackage{amsmath,amssymb}
\usepackage{iftex}
\ifPDFTeX
  \usepackage[T1]{fontenc}
  \usepackage[utf8]{inputenc}
  \usepackage{textcomp} % provide euro and other symbols
\else % if luatex or xetex
  \usepackage{unicode-math} % this also loads fontspec
  \defaultfontfeatures{Scale=MatchLowercase}
  \defaultfontfeatures[\rmfamily]{Ligatures=TeX,Scale=1}
\fi
\usepackage{lmodern}
\ifPDFTeX\else
  % xetex/luatex font selection
\fi
% Use upquote if available, for straight quotes in verbatim environments
\IfFileExists{upquote.sty}{\usepackage{upquote}}{}
\IfFileExists{microtype.sty}{% use microtype if available
  \usepackage[]{microtype}
  \UseMicrotypeSet[protrusion]{basicmath} % disable protrusion for tt fonts
}{}
\makeatletter
\@ifundefined{KOMAClassName}{% if non-KOMA class
  \IfFileExists{parskip.sty}{%
    \usepackage{parskip}
  }{% else
    \setlength{\parindent}{0pt}
    \setlength{\parskip}{6pt plus 2pt minus 1pt}}
}{% if KOMA class
  \KOMAoptions{parskip=half}}
\makeatother
\usepackage{xcolor}
\usepackage[margin=1in]{geometry}
\usepackage{color}
\usepackage{fancyvrb}
\newcommand{\VerbBar}{|}
\newcommand{\VERB}{\Verb[commandchars=\\\{\}]}
\DefineVerbatimEnvironment{Highlighting}{Verbatim}{commandchars=\\\{\}}
% Add ',fontsize=\small' for more characters per line
\usepackage{framed}
\definecolor{shadecolor}{RGB}{248,248,248}
\newenvironment{Shaded}{\begin{snugshade}}{\end{snugshade}}
\newcommand{\AlertTok}[1]{\textcolor[rgb]{0.94,0.16,0.16}{#1}}
\newcommand{\AnnotationTok}[1]{\textcolor[rgb]{0.56,0.35,0.01}{\textbf{\textit{#1}}}}
\newcommand{\AttributeTok}[1]{\textcolor[rgb]{0.13,0.29,0.53}{#1}}
\newcommand{\BaseNTok}[1]{\textcolor[rgb]{0.00,0.00,0.81}{#1}}
\newcommand{\BuiltInTok}[1]{#1}
\newcommand{\CharTok}[1]{\textcolor[rgb]{0.31,0.60,0.02}{#1}}
\newcommand{\CommentTok}[1]{\textcolor[rgb]{0.56,0.35,0.01}{\textit{#1}}}
\newcommand{\CommentVarTok}[1]{\textcolor[rgb]{0.56,0.35,0.01}{\textbf{\textit{#1}}}}
\newcommand{\ConstantTok}[1]{\textcolor[rgb]{0.56,0.35,0.01}{#1}}
\newcommand{\ControlFlowTok}[1]{\textcolor[rgb]{0.13,0.29,0.53}{\textbf{#1}}}
\newcommand{\DataTypeTok}[1]{\textcolor[rgb]{0.13,0.29,0.53}{#1}}
\newcommand{\DecValTok}[1]{\textcolor[rgb]{0.00,0.00,0.81}{#1}}
\newcommand{\DocumentationTok}[1]{\textcolor[rgb]{0.56,0.35,0.01}{\textbf{\textit{#1}}}}
\newcommand{\ErrorTok}[1]{\textcolor[rgb]{0.64,0.00,0.00}{\textbf{#1}}}
\newcommand{\ExtensionTok}[1]{#1}
\newcommand{\FloatTok}[1]{\textcolor[rgb]{0.00,0.00,0.81}{#1}}
\newcommand{\FunctionTok}[1]{\textcolor[rgb]{0.13,0.29,0.53}{\textbf{#1}}}
\newcommand{\ImportTok}[1]{#1}
\newcommand{\InformationTok}[1]{\textcolor[rgb]{0.56,0.35,0.01}{\textbf{\textit{#1}}}}
\newcommand{\KeywordTok}[1]{\textcolor[rgb]{0.13,0.29,0.53}{\textbf{#1}}}
\newcommand{\NormalTok}[1]{#1}
\newcommand{\OperatorTok}[1]{\textcolor[rgb]{0.81,0.36,0.00}{\textbf{#1}}}
\newcommand{\OtherTok}[1]{\textcolor[rgb]{0.56,0.35,0.01}{#1}}
\newcommand{\PreprocessorTok}[1]{\textcolor[rgb]{0.56,0.35,0.01}{\textit{#1}}}
\newcommand{\RegionMarkerTok}[1]{#1}
\newcommand{\SpecialCharTok}[1]{\textcolor[rgb]{0.81,0.36,0.00}{\textbf{#1}}}
\newcommand{\SpecialStringTok}[1]{\textcolor[rgb]{0.31,0.60,0.02}{#1}}
\newcommand{\StringTok}[1]{\textcolor[rgb]{0.31,0.60,0.02}{#1}}
\newcommand{\VariableTok}[1]{\textcolor[rgb]{0.00,0.00,0.00}{#1}}
\newcommand{\VerbatimStringTok}[1]{\textcolor[rgb]{0.31,0.60,0.02}{#1}}
\newcommand{\WarningTok}[1]{\textcolor[rgb]{0.56,0.35,0.01}{\textbf{\textit{#1}}}}
\usepackage{graphicx}
\makeatletter
\def\maxwidth{\ifdim\Gin@nat@width>\linewidth\linewidth\else\Gin@nat@width\fi}
\def\maxheight{\ifdim\Gin@nat@height>\textheight\textheight\else\Gin@nat@height\fi}
\makeatother
% Scale images if necessary, so that they will not overflow the page
% margins by default, and it is still possible to overwrite the defaults
% using explicit options in \includegraphics[width, height, ...]{}
\setkeys{Gin}{width=\maxwidth,height=\maxheight,keepaspectratio}
% Set default figure placement to htbp
\makeatletter
\def\fps@figure{htbp}
\makeatother
\setlength{\emergencystretch}{3em} % prevent overfull lines
\providecommand{\tightlist}{%
  \setlength{\itemsep}{0pt}\setlength{\parskip}{0pt}}
\setcounter{secnumdepth}{-\maxdimen} % remove section numbering
\ifLuaTeX
  \usepackage{selnolig}  % disable illegal ligatures
\fi
\IfFileExists{bookmark.sty}{\usepackage{bookmark}}{\usepackage{hyperref}}
\IfFileExists{xurl.sty}{\usepackage{xurl}}{} % add URL line breaks if available
\urlstyle{same}
\hypersetup{
  pdftitle={R Notebook},
  hidelinks,
  pdfcreator={LaTeX via pandoc}}

\title{R Notebook}
\author{}
\date{\vspace{-2.5em}}

\begin{document}
\maketitle

\hypertarget{lista-de-anuxe1lise-exploratoria}{%
\subsection{Lista de análise
exploratoria}\label{lista-de-anuxe1lise-exploratoria}}

\hypertarget{questuxf5es-8-em-diante}{%
\subsubsection{Questões 8 em diante}\label{questuxf5es-8-em-diante}}

\begin{enumerate}
\def\labelenumi{\arabic{enumi}.}
\setcounter{enumi}{7}
\tightlist
\item
  Um administrador de rede decidiu medir o tempo de carga da página
  principal de um portal Internet; os tempos observados, em
  milissegundos, estão no arquivo de dados globo.dat.
\end{enumerate}

\begin{enumerate}
\def\labelenumi{(\alph{enumi})}
\tightlist
\item
  Qual o tempo típico de carga da página?
\item
  Espera-se que o tempo de carga seja inferior a 600 ms em pelo menos
  98\% dos casos. Isso está ocorrendo na prática
\end{enumerate}

\begin{Shaded}
\begin{Highlighting}[]
\NormalTok{globo\_data }\OtherTok{\textless{}{-}} \FunctionTok{read.table}\NormalTok{(}\StringTok{"./asc{-}dados/globo.dat"}\NormalTok{)}
\end{Highlighting}
\end{Shaded}

\begin{Shaded}
\begin{Highlighting}[]
\FunctionTok{boxplot}\NormalTok{(globo\_data}\SpecialCharTok{$}\NormalTok{V1)}
\end{Highlighting}
\end{Shaded}

\includegraphics{exercicios_exploratoria_files/figure-latex/unnamed-chunk-2-1.pdf}

\begin{Shaded}
\begin{Highlighting}[]
\NormalTok{(globo\_data}\SpecialCharTok{$}\NormalTok{V1[globo\_data}\SpecialCharTok{$}\NormalTok{V1 }\SpecialCharTok{\textgreater{}} \DecValTok{1000}\NormalTok{])}
\end{Highlighting}
\end{Shaded}

\begin{verbatim}
## [1] 38454 19597
\end{verbatim}

\begin{Shaded}
\begin{Highlighting}[]
\NormalTok{globo2 }\OtherTok{\textless{}{-}}\NormalTok{ globo\_data[}\SpecialCharTok{!}\NormalTok{(globo\_data }\SpecialCharTok{==} \DecValTok{0} \SpecialCharTok{|}\NormalTok{ globo\_data }\SpecialCharTok{\textgreater{}} \DecValTok{1000}\NormalTok{)]}
\FunctionTok{hist}\NormalTok{(globo2)}
\end{Highlighting}
\end{Shaded}

\includegraphics{exercicios_exploratoria_files/figure-latex/unnamed-chunk-3-1.pdf}

\begin{Shaded}
\begin{Highlighting}[]
\FunctionTok{summary}\NormalTok{(globo2)}
\end{Highlighting}
\end{Shaded}

\begin{verbatim}
##    Min. 1st Qu.  Median    Mean 3rd Qu.    Max. 
##   143.0   493.8   512.0   516.5   532.2   838.0
\end{verbatim}

\begin{Shaded}
\begin{Highlighting}[]
\NormalTok{globo3 }\OtherTok{\textless{}{-}}\NormalTok{ globo\_data[globo\_data }\SpecialCharTok{!=} \DecValTok{0}\NormalTok{]}
\FunctionTok{sum}\NormalTok{(globo3 }\SpecialCharTok{\textless{}} \DecValTok{600}\NormalTok{) }\SpecialCharTok{/} \FunctionTok{length}\NormalTok{(globo3) }
\end{Highlighting}
\end{Shaded}

\begin{verbatim}
## [1] 0.9738956
\end{verbatim}

\begin{Shaded}
\begin{Highlighting}[]
\CommentTok{\# Quase 98\% da requisções ficaram abaixo de 600ms porem quase não é ficar rs;}
\end{Highlighting}
\end{Shaded}

Exercicio 9.

\begin{Shaded}
\begin{Highlighting}[]
\NormalTok{pacotes }\OtherTok{\textless{}{-}} \FunctionTok{read.table}\NormalTok{(}\StringTok{"./asc{-}dados/pacotes.dat"}\NormalTok{, }\AttributeTok{head=}\NormalTok{T)}
\FunctionTok{summary}\NormalTok{(pacotes}\SpecialCharTok{$}\NormalTok{tam)}
\end{Highlighting}
\end{Shaded}

\begin{verbatim}
##    Min. 1st Qu.  Median    Mean 3rd Qu.    Max. 
##    40.0    52.0  1500.0   947.3  1500.0  1500.0
\end{verbatim}

\begin{Shaded}
\begin{Highlighting}[]
\FunctionTok{hist}\NormalTok{(pacotes}\SpecialCharTok{$}\NormalTok{tam, }\AttributeTok{breaks =} \DecValTok{100}\NormalTok{)}
\end{Highlighting}
\end{Shaded}

\includegraphics{exercicios_exploratoria_files/figure-latex/unnamed-chunk-5-1.pdf}

\begin{Shaded}
\begin{Highlighting}[]
\FunctionTok{boxplot}\NormalTok{(pacotes}\SpecialCharTok{$}\NormalTok{tam)}
\end{Highlighting}
\end{Shaded}

\includegraphics{exercicios_exploratoria_files/figure-latex/unnamed-chunk-5-2.pdf}

\begin{Shaded}
\begin{Highlighting}[]
\FunctionTok{plot}\NormalTok{(pacotes}\SpecialCharTok{$}\NormalTok{tam)}
\end{Highlighting}
\end{Shaded}

\includegraphics{exercicios_exploratoria_files/figure-latex/unnamed-chunk-5-3.pdf}

\begin{Shaded}
\begin{Highlighting}[]
\FunctionTok{sum}\NormalTok{(pacotes}\SpecialCharTok{$}\NormalTok{tam }\SpecialCharTok{\textgreater{}} \DecValTok{1000}\NormalTok{) }\SpecialCharTok{/} \FunctionTok{length}\NormalTok{(pacotes}\SpecialCharTok{$}\NormalTok{tam)}
\end{Highlighting}
\end{Shaded}

\begin{verbatim}
## [1] 0.6141
\end{verbatim}

\begin{Shaded}
\begin{Highlighting}[]
\FunctionTok{sum}\NormalTok{(pacotes}\SpecialCharTok{$}\NormalTok{tam }\SpecialCharTok{\textless{}} \DecValTok{100}\NormalTok{) }\SpecialCharTok{/} \FunctionTok{length}\NormalTok{(pacotes}\SpecialCharTok{$}\NormalTok{tam)}
\end{Highlighting}
\end{Shaded}

\begin{verbatim}
## [1] 0.3028
\end{verbatim}

\begin{Shaded}
\begin{Highlighting}[]
\CommentTok{\# Dificil usar média, moda ou mediana neste caso, os dados estão distribuidos nas pontas (30\% dos dados são menores ou iguas a 100ms e 61\% são maiores que 1000ms) oque dificulda a analise.}
\end{Highlighting}
\end{Shaded}

\begin{Shaded}
\begin{Highlighting}[]
\FunctionTok{table}\NormalTok{(pacotes}\SpecialCharTok{$}\NormalTok{tam)}
\end{Highlighting}
\end{Shaded}

\begin{verbatim}
## 
##   40   50   52   57   60   64   66   72   78   80   86   89   90   92   93   96 
##    7    4 2814    6    1   18    1   14   92   41    2    6    2    3    8    8 
##   99  105  109  111  112  118  122  123  125  126  138  142  143  153  158  163 
##    1   15   11    3    2   63    3   55  563   15    1    5    2    3    4    2 
##  169  174  209  215  217  222  227  229  240  248  249  256  263  273  276  277 
##    2    4    4    4    5    3    6    1    1    1    1    1    2    8    1    1 
##  290  300  326  340  411  459  468  505  506  515  517  520  548  583  756  945 
##    1    8    2    1    2    1    2    2    2    1    2    2    8    1    2    1 
##  955 1004 1044 1124 1186 1252 1322 1332 1460 1470 1500 
##    1    3    4    1    1   74    1    1    1    1 6054
\end{verbatim}

\begin{Shaded}
\begin{Highlighting}[]
\FunctionTok{sort}\NormalTok{(}\FunctionTok{table}\NormalTok{(pacotes}\SpecialCharTok{$}\NormalTok{tam), }\AttributeTok{decreasing =}\NormalTok{ T)}
\end{Highlighting}
\end{Shaded}

\begin{verbatim}
## 
## 1500   52  125   78 1252  118  123   80   64  105  126   72  109   93   96  273 
## 6054 2814  563   92   74   63   55   41   18   15   15   14   11    8    8    8 
##  300  548   40   57   89  227  142  217   50  158  174  209  215 1044   92  111 
##    8    8    7    6    6    6    5    5    4    4    4    4    4    4    3    3 
##  122  153  222 1004   86   90  112  143  163  169  263  326  411  468  505  506 
##    3    3    3    3    2    2    2    2    2    2    2    2    2    2    2    2 
##  517  520  756   60   66   99  138  229  240  248  249  256  276  277  290  340 
##    2    2    2    1    1    1    1    1    1    1    1    1    1    1    1    1 
##  459  515  583  945  955 1124 1186 1322 1332 1460 1470 
##    1    1    1    1    1    1    1    1    1    1    1
\end{verbatim}

Exercicio 10

\begin{Shaded}
\begin{Highlighting}[]
\NormalTok{sgbd }\OtherTok{\textless{}{-}} \FunctionTok{read.table}\NormalTok{(}\StringTok{"./asc{-}dados/trbd.dat"}\NormalTok{, }\AttributeTok{head=}\NormalTok{T)}
\FunctionTok{head}\NormalTok{(sgbd)}
\end{Highlighting}
\end{Shaded}

\begin{verbatim}
##   tr.bd
## 1    83
## 2    36
## 3    35
## 4    37
## 5    47
## 6    74
\end{verbatim}

\begin{Shaded}
\begin{Highlighting}[]
\FunctionTok{hist}\NormalTok{(sgbd}\SpecialCharTok{$}\NormalTok{tr.bd)}
\end{Highlighting}
\end{Shaded}

\includegraphics{exercicios_exploratoria_files/figure-latex/unnamed-chunk-7-1.pdf}

\begin{Shaded}
\begin{Highlighting}[]
\FunctionTok{summary}\NormalTok{(sgbd}\SpecialCharTok{$}\NormalTok{tr.bd)}
\end{Highlighting}
\end{Shaded}

\begin{verbatim}
##    Min. 1st Qu.  Median    Mean 3rd Qu.    Max. 
##   20.00   35.00   44.50   47.15   57.00   95.00
\end{verbatim}

\begin{Shaded}
\begin{Highlighting}[]
\NormalTok{CV }\OtherTok{\textless{}{-}} \ControlFlowTok{function}\NormalTok{(x) \{ }\FunctionTok{sd}\NormalTok{(x) }\SpecialCharTok{/} \FunctionTok{mean}\NormalTok{(x) \} }\CommentTok{\# Coefiecente de variação}

\FunctionTok{CV}\NormalTok{(sgbd}\SpecialCharTok{$}\NormalTok{tr.bd)}
\end{Highlighting}
\end{Shaded}

\begin{verbatim}
## [1] 0.3452368
\end{verbatim}

\begin{Shaded}
\begin{Highlighting}[]
\CommentTok{\# Vazão maxima será quando a resposta for minima}
\CommentTok{\# 50 consultas por segundo, sendo a consulta minima 20ms}
\end{Highlighting}
\end{Shaded}

Exercicio 11

\begin{Shaded}
\begin{Highlighting}[]
\NormalTok{users }\OtherTok{\textless{}{-}} \FunctionTok{read.table}\NormalTok{(}\StringTok{"./asc{-}dados/usuarios.dat"}\NormalTok{, }\AttributeTok{head=}\NormalTok{T)}

\FunctionTok{summary}\NormalTok{(users)}
\end{Highlighting}
\end{Shaded}

\begin{verbatim}
##    horario              users      
##  Length:600         Min.   : 51.0  
##  Class :character   1st Qu.:172.8  
##  Mode  :character   Median :201.0  
##                     Mean   :190.3  
##                     3rd Qu.:219.2  
##                     Max.   :269.0
\end{verbatim}

\begin{Shaded}
\begin{Highlighting}[]
\FunctionTok{hist}\NormalTok{(users}\SpecialCharTok{$}\NormalTok{users)}
\end{Highlighting}
\end{Shaded}

\includegraphics{exercicios_exploratoria_files/figure-latex/unnamed-chunk-8-1.pdf}

\begin{Shaded}
\begin{Highlighting}[]
\CommentTok{\# Min 51.0 }
\CommentTok{\# Max 269.0}

\CommentTok{\# Horarios}

\CommentTok{\# pode tbm usar which com o dataframe}
\NormalTok{(users}\SpecialCharTok{$}\NormalTok{horario[users}\SpecialCharTok{$}\NormalTok{users }\SpecialCharTok{==} \FunctionTok{max}\NormalTok{(users}\SpecialCharTok{$}\NormalTok{users)])}
\end{Highlighting}
\end{Shaded}

\begin{verbatim}
## [1] "16:45"
\end{verbatim}

\begin{Shaded}
\begin{Highlighting}[]
\NormalTok{(users}\SpecialCharTok{$}\NormalTok{horario[users}\SpecialCharTok{$}\NormalTok{users }\SpecialCharTok{==} \FunctionTok{min}\NormalTok{(users}\SpecialCharTok{$}\NormalTok{users)])}
\end{Highlighting}
\end{Shaded}

\begin{verbatim}
## [1] "12:20"
\end{verbatim}

\begin{Shaded}
\begin{Highlighting}[]
\CommentTok{\# B) Divide em 3 periodos e pega media,mediana {-}\textgreater{} faz sentido}
\CommentTok{\# Numero tipico de usuarios = mediana = 201.0}

\CommentTok{\# Horarios acima do medido na analise}
\NormalTok{(users}\SpecialCharTok{$}\NormalTok{horario[users}\SpecialCharTok{$}\NormalTok{users }\SpecialCharTok{\textgreater{}} \DecValTok{260}\NormalTok{])}
\end{Highlighting}
\end{Shaded}

\begin{verbatim}
## [1] "09:11" "14:44" "14:45" "14:57" "15:33" "15:47" "16:45" "16:50"
\end{verbatim}

\begin{Shaded}
\begin{Highlighting}[]
\CommentTok{\# Assimetria, lado que está a cauda}

\CommentTok{\# Plota o dataframe}
\FunctionTok{plot}\NormalTok{(users}\SpecialCharTok{$}\NormalTok{users, }\AttributeTok{xaxt=}\StringTok{"n"}\NormalTok{)}
\FunctionTok{axis}\NormalTok{(}\DecValTok{1}\NormalTok{, }\AttributeTok{at =}\DecValTok{1}\SpecialCharTok{:}\DecValTok{600}\NormalTok{, }\AttributeTok{labels=}\NormalTok{users}\SpecialCharTok{$}\NormalTok{horario)}
\FunctionTok{abline}\NormalTok{(}\AttributeTok{v=}\FunctionTok{c}\NormalTok{(}\DecValTok{241}\NormalTok{,}\DecValTok{360}\NormalTok{), }\AttributeTok{lty=}\DecValTok{2}\NormalTok{)}
\end{Highlighting}
\end{Shaded}

\includegraphics{exercicios_exploratoria_files/figure-latex/unnamed-chunk-8-2.pdf}

\begin{Shaded}
\begin{Highlighting}[]
\NormalTok{users[}\FunctionTok{which}\NormalTok{(users}\SpecialCharTok{$}\NormalTok{users }\SpecialCharTok{\textgreater{}} \DecValTok{260}\NormalTok{), ]}
\end{Highlighting}
\end{Shaded}

\begin{verbatim}
##     horario users
## 71    09:11   261
## 404   14:44   262
## 405   14:45   266
## 417   14:57   262
## 453   15:33   263
## 467   15:47   264
## 525   16:45   269
## 530   16:50   263
\end{verbatim}

\hfill\break
Exer 12.

\begin{Shaded}
\begin{Highlighting}[]
\NormalTok{tmp }\OtherTok{\textless{}{-}} \FunctionTok{read.table}\NormalTok{(}\StringTok{"./asc{-}dados/aed{-}tcons.dat"}\NormalTok{, }\AttributeTok{head=}\NormalTok{T)}
\FunctionTok{hist}\NormalTok{(tmp}\SpecialCharTok{$}\NormalTok{S1) }\CommentTok{\# Assimemtrica a direita}
\end{Highlighting}
\end{Shaded}

\includegraphics{exercicios_exploratoria_files/figure-latex/unnamed-chunk-9-1.pdf}

\begin{Shaded}
\begin{Highlighting}[]
\FunctionTok{hist}\NormalTok{(tmp}\SpecialCharTok{$}\NormalTok{S2) }\CommentTok{\# bimodal simetrica}
\end{Highlighting}
\end{Shaded}

\includegraphics{exercicios_exploratoria_files/figure-latex/unnamed-chunk-9-2.pdf}

\begin{Shaded}
\begin{Highlighting}[]
\FunctionTok{hist}\NormalTok{(tmp}\SpecialCharTok{$}\NormalTok{S3) }\CommentTok{\# Simetrica}
\end{Highlighting}
\end{Shaded}

\includegraphics{exercicios_exploratoria_files/figure-latex/unnamed-chunk-9-3.pdf}

\begin{Shaded}
\begin{Highlighting}[]
\FunctionTok{hist}\NormalTok{(tmp}\SpecialCharTok{$}\NormalTok{S4) }\CommentTok{\# Assimetrica a esquerda}
\end{Highlighting}
\end{Shaded}

\includegraphics{exercicios_exploratoria_files/figure-latex/unnamed-chunk-9-4.pdf}

\begin{Shaded}
\begin{Highlighting}[]
\FunctionTok{summary}\NormalTok{(tmp}\SpecialCharTok{$}\NormalTok{S3)}
\end{Highlighting}
\end{Shaded}

\begin{verbatim}
##    Min. 1st Qu.  Median    Mean 3rd Qu.    Max. 
##   16.94   43.02   50.65   50.05   56.51   89.86
\end{verbatim}

\begin{Shaded}
\begin{Highlighting}[]
\FunctionTok{print}\NormalTok{(}\StringTok{"{-}{-}"}\NormalTok{)}
\end{Highlighting}
\end{Shaded}

\begin{verbatim}
## [1] "--"
\end{verbatim}

\begin{Shaded}
\begin{Highlighting}[]
\FunctionTok{summary}\NormalTok{(tmp}\SpecialCharTok{$}\NormalTok{S2)}
\end{Highlighting}
\end{Shaded}

\begin{verbatim}
##    Min. 1st Qu.  Median    Mean 3rd Qu.    Max. 
##   7.383  20.386  34.502  35.205  50.080  64.402
\end{verbatim}

\begin{Shaded}
\begin{Highlighting}[]
\CommentTok{\# b) Calcule o 3o quartil, o 90o e o 95o percentis dos tempos de consulta de S4.}

\CommentTok{\# summary(tmp$S4) \#  3rd Quartil: 69.89}
\FunctionTok{quantile}\NormalTok{(tmp}\SpecialCharTok{$}\NormalTok{S4, }\FloatTok{0.75}\NormalTok{)}
\end{Highlighting}
\end{Shaded}

\begin{verbatim}
##    75% 
## 69.886
\end{verbatim}

\begin{Shaded}
\begin{Highlighting}[]
\FunctionTok{quantile}\NormalTok{(tmp}\SpecialCharTok{$}\NormalTok{S4, }\FloatTok{0.90}\NormalTok{)}
\end{Highlighting}
\end{Shaded}

\begin{verbatim}
##      90% 
## 100.1858
\end{verbatim}

\begin{Shaded}
\begin{Highlighting}[]
\FunctionTok{quantile}\NormalTok{(tmp}\SpecialCharTok{$}\NormalTok{S4, }\FloatTok{0.95}\NormalTok{)}
\end{Highlighting}
\end{Shaded}

\begin{verbatim}
##      95% 
## 116.7008
\end{verbatim}

\begin{Shaded}
\begin{Highlighting}[]
\CommentTok{\# (c) Calcule a mediana, a média aritmética e média truncada a 5\% para os tempos de consulta de S1 e S3. Como o truncamento afeta a média de cada sistema?}

\FunctionTok{median}\NormalTok{(tmp}\SpecialCharTok{$}\NormalTok{S1)}
\end{Highlighting}
\end{Shaded}

\begin{verbatim}
## [1] 84.749
\end{verbatim}

\begin{Shaded}
\begin{Highlighting}[]
\FunctionTok{mean}\NormalTok{(tmp}\SpecialCharTok{$}\NormalTok{S1)}
\end{Highlighting}
\end{Shaded}

\begin{verbatim}
## [1] 79.64171
\end{verbatim}

\begin{Shaded}
\begin{Highlighting}[]
\FunctionTok{mean}\NormalTok{(tmp}\SpecialCharTok{$}\NormalTok{S1, }\AttributeTok{trim =} \FloatTok{0.05}\NormalTok{)}
\end{Highlighting}
\end{Shaded}

\begin{verbatim}
## [1] 81.03633
\end{verbatim}

\begin{Shaded}
\begin{Highlighting}[]
\FunctionTok{print}\NormalTok{(}\StringTok{"\#\#\#"}\NormalTok{)}
\end{Highlighting}
\end{Shaded}

\begin{verbatim}
## [1] "###"
\end{verbatim}

\begin{Shaded}
\begin{Highlighting}[]
\FunctionTok{median}\NormalTok{(tmp}\SpecialCharTok{$}\NormalTok{S3)}
\end{Highlighting}
\end{Shaded}

\begin{verbatim}
## [1] 50.648
\end{verbatim}

\begin{Shaded}
\begin{Highlighting}[]
\FunctionTok{mean}\NormalTok{(tmp}\SpecialCharTok{$}\NormalTok{S3)}
\end{Highlighting}
\end{Shaded}

\begin{verbatim}
## [1] 50.05369
\end{verbatim}

\begin{Shaded}
\begin{Highlighting}[]
\FunctionTok{mean}\NormalTok{(tmp}\SpecialCharTok{$}\NormalTok{S3, }\AttributeTok{trim =} \FloatTok{0.05}\NormalTok{)}
\end{Highlighting}
\end{Shaded}

\begin{verbatim}
## [1] 50.10067
\end{verbatim}

\begin{Shaded}
\begin{Highlighting}[]
\CommentTok{\# Aumenta a média, visto que o valores (5\%) maiores e menores cairam fora}

\CommentTok{\# D}
\CommentTok{\# ECDF1 {-} Ae}
\CommentTok{\# ECDF2 {-} BM}
\CommentTok{\# ECDF3 {-} US}
\CommentTok{\# ECDF4 {-} AD}
\end{Highlighting}
\end{Shaded}

\begin{Shaded}
\begin{Highlighting}[]
\CommentTok{\# Plota tudo juntinho do lado}
\FunctionTok{par}\NormalTok{()}
\end{Highlighting}
\end{Shaded}

\begin{verbatim}
## $xlog
## [1] FALSE
## 
## $ylog
## [1] FALSE
## 
## $adj
## [1] 0.5
## 
## $ann
## [1] TRUE
## 
## $ask
## [1] FALSE
## 
## $bg
## [1] "transparent"
## 
## $bty
## [1] "o"
## 
## $cex
## [1] 1
## 
## $cex.axis
## [1] 1
## 
## $cex.lab
## [1] 1
## 
## $cex.main
## [1] 1.2
## 
## $cex.sub
## [1] 1
## 
## $cin
## [1] 0.15 0.20
## 
## $col
## [1] "black"
## 
## $col.axis
## [1] "black"
## 
## $col.lab
## [1] "black"
## 
## $col.main
## [1] "black"
## 
## $col.sub
## [1] "black"
## 
## $cra
## [1] 10.8 14.4
## 
## $crt
## [1] 0
## 
## $csi
## [1] 0.2
## 
## $cxy
## [1] 0.02851711 0.07518797
## 
## $din
## [1] 6.5 4.5
## 
## $err
## [1] 0
## 
## $family
## [1] ""
## 
## $fg
## [1] "black"
## 
## $fig
## [1] 0 1 0 1
## 
## $fin
## [1] 6.5 4.5
## 
## $font
## [1] 1
## 
## $font.axis
## [1] 1
## 
## $font.lab
## [1] 1
## 
## $font.main
## [1] 2
## 
## $font.sub
## [1] 1
## 
## $lab
## [1] 5 5 7
## 
## $las
## [1] 0
## 
## $lend
## [1] "round"
## 
## $lheight
## [1] 1
## 
## $ljoin
## [1] "round"
## 
## $lmitre
## [1] 10
## 
## $lty
## [1] "solid"
## 
## $lwd
## [1] 1
## 
## $mai
## [1] 1.02 0.82 0.82 0.42
## 
## $mar
## [1] 5.1 4.1 4.1 2.1
## 
## $mex
## [1] 1
## 
## $mfcol
## [1] 1 1
## 
## $mfg
## [1] 1 1 1 1
## 
## $mfrow
## [1] 1 1
## 
## $mgp
## [1] 3 1 0
## 
## $mkh
## [1] 0.001
## 
## $new
## [1] FALSE
## 
## $oma
## [1] 0 0 0 0
## 
## $omd
## [1] 0 1 0 1
## 
## $omi
## [1] 0 0 0 0
## 
## $page
## [1] TRUE
## 
## $pch
## [1] 1
## 
## $pin
## [1] 5.26 2.66
## 
## $plt
## [1] 0.1261538 0.9353846 0.2266667 0.8177778
## 
## $ps
## [1] 12
## 
## $pty
## [1] "m"
## 
## $smo
## [1] 1
## 
## $srt
## [1] 0
## 
## $tck
## [1] NA
## 
## $tcl
## [1] -0.5
## 
## $usr
## [1] 0 1 0 1
## 
## $xaxp
## [1] 0 1 5
## 
## $xaxs
## [1] "r"
## 
## $xaxt
## [1] "s"
## 
## $xpd
## [1] FALSE
## 
## $yaxp
## [1] 0 1 5
## 
## $yaxs
## [1] "r"
## 
## $yaxt
## [1] "s"
## 
## $ylbias
## [1] 0.2
\end{verbatim}

\end{document}
